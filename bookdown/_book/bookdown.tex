\documentclass[]{book}
\usepackage{lmodern}
\usepackage{amssymb,amsmath}
\usepackage{ifxetex,ifluatex}
\usepackage{fixltx2e} % provides \textsubscript
\ifnum 0\ifxetex 1\fi\ifluatex 1\fi=0 % if pdftex
  \usepackage[T1]{fontenc}
  \usepackage[utf8]{inputenc}
\else % if luatex or xelatex
  \ifxetex
    \usepackage{mathspec}
  \else
    \usepackage{fontspec}
  \fi
  \defaultfontfeatures{Ligatures=TeX,Scale=MatchLowercase}
\fi
% use upquote if available, for straight quotes in verbatim environments
\IfFileExists{upquote.sty}{\usepackage{upquote}}{}
% use microtype if available
\IfFileExists{microtype.sty}{%
\usepackage{microtype}
\UseMicrotypeSet[protrusion]{basicmath} % disable protrusion for tt fonts
}{}
\usepackage{hyperref}
\hypersetup{unicode=true,
            pdftitle={Notas de clase del curso de introducción a Data Science},
            pdfauthor={Diego Kozlowski y Natsumi Shokida},
            pdfborder={0 0 0},
            breaklinks=true}
\urlstyle{same}  % don't use monospace font for urls
\usepackage{natbib}
\bibliographystyle{apalike}
\usepackage{color}
\usepackage{fancyvrb}
\newcommand{\VerbBar}{|}
\newcommand{\VERB}{\Verb[commandchars=\\\{\}]}
\DefineVerbatimEnvironment{Highlighting}{Verbatim}{commandchars=\\\{\}}
% Add ',fontsize=\small' for more characters per line
\usepackage{framed}
\definecolor{shadecolor}{RGB}{248,248,248}
\newenvironment{Shaded}{\begin{snugshade}}{\end{snugshade}}
\newcommand{\AlertTok}[1]{\textcolor[rgb]{0.94,0.16,0.16}{#1}}
\newcommand{\AnnotationTok}[1]{\textcolor[rgb]{0.56,0.35,0.01}{\textbf{\textit{#1}}}}
\newcommand{\AttributeTok}[1]{\textcolor[rgb]{0.77,0.63,0.00}{#1}}
\newcommand{\BaseNTok}[1]{\textcolor[rgb]{0.00,0.00,0.81}{#1}}
\newcommand{\BuiltInTok}[1]{#1}
\newcommand{\CharTok}[1]{\textcolor[rgb]{0.31,0.60,0.02}{#1}}
\newcommand{\CommentTok}[1]{\textcolor[rgb]{0.56,0.35,0.01}{\textit{#1}}}
\newcommand{\CommentVarTok}[1]{\textcolor[rgb]{0.56,0.35,0.01}{\textbf{\textit{#1}}}}
\newcommand{\ConstantTok}[1]{\textcolor[rgb]{0.00,0.00,0.00}{#1}}
\newcommand{\ControlFlowTok}[1]{\textcolor[rgb]{0.13,0.29,0.53}{\textbf{#1}}}
\newcommand{\DataTypeTok}[1]{\textcolor[rgb]{0.13,0.29,0.53}{#1}}
\newcommand{\DecValTok}[1]{\textcolor[rgb]{0.00,0.00,0.81}{#1}}
\newcommand{\DocumentationTok}[1]{\textcolor[rgb]{0.56,0.35,0.01}{\textbf{\textit{#1}}}}
\newcommand{\ErrorTok}[1]{\textcolor[rgb]{0.64,0.00,0.00}{\textbf{#1}}}
\newcommand{\ExtensionTok}[1]{#1}
\newcommand{\FloatTok}[1]{\textcolor[rgb]{0.00,0.00,0.81}{#1}}
\newcommand{\FunctionTok}[1]{\textcolor[rgb]{0.00,0.00,0.00}{#1}}
\newcommand{\ImportTok}[1]{#1}
\newcommand{\InformationTok}[1]{\textcolor[rgb]{0.56,0.35,0.01}{\textbf{\textit{#1}}}}
\newcommand{\KeywordTok}[1]{\textcolor[rgb]{0.13,0.29,0.53}{\textbf{#1}}}
\newcommand{\NormalTok}[1]{#1}
\newcommand{\OperatorTok}[1]{\textcolor[rgb]{0.81,0.36,0.00}{\textbf{#1}}}
\newcommand{\OtherTok}[1]{\textcolor[rgb]{0.56,0.35,0.01}{#1}}
\newcommand{\PreprocessorTok}[1]{\textcolor[rgb]{0.56,0.35,0.01}{\textit{#1}}}
\newcommand{\RegionMarkerTok}[1]{#1}
\newcommand{\SpecialCharTok}[1]{\textcolor[rgb]{0.00,0.00,0.00}{#1}}
\newcommand{\SpecialStringTok}[1]{\textcolor[rgb]{0.31,0.60,0.02}{#1}}
\newcommand{\StringTok}[1]{\textcolor[rgb]{0.31,0.60,0.02}{#1}}
\newcommand{\VariableTok}[1]{\textcolor[rgb]{0.00,0.00,0.00}{#1}}
\newcommand{\VerbatimStringTok}[1]{\textcolor[rgb]{0.31,0.60,0.02}{#1}}
\newcommand{\WarningTok}[1]{\textcolor[rgb]{0.56,0.35,0.01}{\textbf{\textit{#1}}}}
\usepackage{longtable,booktabs}
\usepackage{graphicx,grffile}
\makeatletter
\def\maxwidth{\ifdim\Gin@nat@width>\linewidth\linewidth\else\Gin@nat@width\fi}
\def\maxheight{\ifdim\Gin@nat@height>\textheight\textheight\else\Gin@nat@height\fi}
\makeatother
% Scale images if necessary, so that they will not overflow the page
% margins by default, and it is still possible to overwrite the defaults
% using explicit options in \includegraphics[width, height, ...]{}
\setkeys{Gin}{width=\maxwidth,height=\maxheight,keepaspectratio}
\IfFileExists{parskip.sty}{%
\usepackage{parskip}
}{% else
\setlength{\parindent}{0pt}
\setlength{\parskip}{6pt plus 2pt minus 1pt}
}
\setlength{\emergencystretch}{3em}  % prevent overfull lines
\providecommand{\tightlist}{%
  \setlength{\itemsep}{0pt}\setlength{\parskip}{0pt}}
\setcounter{secnumdepth}{5}
% Redefines (sub)paragraphs to behave more like sections
\ifx\paragraph\undefined\else
\let\oldparagraph\paragraph
\renewcommand{\paragraph}[1]{\oldparagraph{#1}\mbox{}}
\fi
\ifx\subparagraph\undefined\else
\let\oldsubparagraph\subparagraph
\renewcommand{\subparagraph}[1]{\oldsubparagraph{#1}\mbox{}}
\fi

%%% Use protect on footnotes to avoid problems with footnotes in titles
\let\rmarkdownfootnote\footnote%
\def\footnote{\protect\rmarkdownfootnote}

%%% Change title format to be more compact
\usepackage{titling}

% Create subtitle command for use in maketitle
\providecommand{\subtitle}[1]{
  \posttitle{
    \begin{center}\large#1\end{center}
    }
}

\setlength{\droptitle}{-2em}

  \title{Notas de clase del curso de introducción a Data Science}
    \pretitle{\vspace{\droptitle}\centering\huge}
  \posttitle{\par}
    \author{Diego Kozlowski y Natsumi Shokida}
    \preauthor{\centering\large\emph}
  \postauthor{\par}
      \predate{\centering\large\emph}
  \postdate{\par}
    \date{2019-08-20}

\usepackage{booktabs}

\begin{document}
\maketitle

{
\setcounter{tocdepth}{1}
\tableofcontents
}
\hypertarget{temario}{%
\chapter{Temario}\label{temario}}

\begin{itemize}
\tightlist
\item
  \textbf{clase 1}: Introducción al entorno R
\item
  \textbf{clase 2}: Tidyverse. (limpieza y organización de datos)
\item
  \textbf{clase 3}: Visualización de la información
\item
  \textbf{clase 4}: Estadística descriptiva
\item
  \textbf{clase 5}:
\item
  \textbf{clase 6}:
\item
  \textbf{clase 7}:
\item
  \textbf{clase 8}:
\item
  \textbf{clase 9}:
\item
  \textbf{clase 10}:
\end{itemize}

\hypertarget{librerias-a-instalar}{%
\subsubsection{Librerias a instalar}\label{librerias-a-instalar}}

\begin{verbatim}
install.packages(c("tidyverse","openxlsx",'ggplot2','ggthemes', 'ggrepel','ggalt','kableExtra','GGally','ggridges','fs','purrr','rmarkdown'))
\end{verbatim}

\hypertarget{introduccion-a-r}{%
\chapter{Introducción a R}\label{introduccion-a-r}}

\hypertarget{explicacion}{%
\section{Explicación}\label{explicacion}}

\begin{figure}
\centering
\includegraphics{img/Rlogo.png}
\caption{\url{https://cran.r-project.org/}}
\end{figure}

\hypertarget{que-es-r}{%
\subsection{¿Que es R?}\label{que-es-r}}

\begin{itemize}
\tightlist
\item
  Lenguaje para el procesamiento y análisis estadístico de datos
\item
  Software Libre
\item
  Sintaxis Básica: R base
\item
  Sintaxis incremental\footnote{Más allá de los comandos elementales, comandos más sofisticados tienen muchas versiones, y algunas quedan en desuso en el tiempo.}: El lenguaje se va ampliando por aportes de Universidades, investigadores/as, usuarios/as y empresas privadas, organizados en librerías (o paquetes)
\item
  Comunidad web muy grande para realizar preguntas y despejar dudas.
\item
  Graficos con calidad de publicación
\end{itemize}

\begin{figure}
\centering
\includegraphics{img/number-of-submitted-packages-to-CRAN.png}
\caption{fuente: \url{https://gist.github.com/daroczig/3cf06d6db4be2bbe3368}}
\end{figure}

\begin{figure}
\centering
\includegraphics{img/RStudiologo.png}
\caption{\url{https://www.rstudio.com/}}
\end{figure}

El \emph{entorno} más cómodo para utilizar el \emph{lenguaje} \textbf{R} es el \emph{programa} \textbf{R studio}

\begin{itemize}
\item
  Rstudio es una empresa que produce productos asociados al lenguaje R, como el programa sobre el que corremos los comandos, y extensiones del lenguaje (librerías).
\item
  El programa es \emph{gratuito} y se puede bajar de la
  \href{https://www.rstudio.com/}{página oficial}
\end{itemize}

\begin{figure}
\centering
\includegraphics{img/Pantalla Rstudio.png}
\caption{Pantalla Rstudio}
\end{figure}

\hypertarget{diferencias-con-stata-y-spss}{%
\subsection{Diferencias con STATA y SPSS}\label{diferencias-con-stata-y-spss}}

\begin{itemize}
\tightlist
\item
  Gratuito
\item
  Funciona principalmente por líneas de código (Aunque ya hay paquetes que permiten ejecutar comandos desde el menú y los botones sin tener que esribir código)
\item
  Trabaja las bases de microdatos de forma virtual y no fisica, lo que permite disponer de varias al mismo tiempo sin mayor dificultad (no requiere abrir cada base, trabajarla por separado y luego cerrarla), ni exije guardar fisicamente los cambios.
\item
  Más potente

  \begin{itemize}
  \tightlist
  \item
    Totalmente automatizable
  \item
    Aportes de usuarias y usuarios
  \item
    Extensible a otros lenguajes y usos (presentación como esta, diseño de aplicaciones)
  \end{itemize}
\item
  Más veloz:
\end{itemize}

\begin{figure}
\centering
\includegraphics{img/stataR.png}
\caption{fuente: \url{https://github.com/matthieugomez/benchmark-stata-r/blob/master/output/1e7.png}}
\end{figure}

\hypertarget{logica-sintactica.}{%
\subsection{Lógica sintáctica.}\label{logica-sintactica.}}

\hypertarget{definicion-de-objetos}{%
\subsubsection{Definición de objetos}\label{definicion-de-objetos}}

Los \textbf{Objetos/Elementos} constituyen la categoría escencial del R. De hecho, todo en R es un objeto, y se almacena con un nombre específico que \textbf{no debe poseer espacios}. Un número, un vector, una función, la progresión de letras del abecedario, una base de datos, un gráfico, constituyen para R objetos de distinto tipo. Los objetos que vamos creando a medida que trabajamos pueden visualizarse en la panel derecho superior de la pantalla.

El operador \textbf{\texttt{\textless{}-}} sirve para definir un objeto. \textbf{A la izquierda} del \textbf{\texttt{\textless{}-}} debe ubicarse el nombre que tomará el elemento a crear. \textbf{Del lado derecho} debe ir la definición del mismo

\begin{Shaded}
\begin{Highlighting}[]
\NormalTok{A <-}\StringTok{ }\DecValTok{1}
\end{Highlighting}
\end{Shaded}

Al definir un elemento, el mismo queda guardado en el ambiente del programa, y podrá ser utilizado posteriormente para observar su contenido o para realizar una operación con el mismo

\begin{Shaded}
\begin{Highlighting}[]
\NormalTok{A }
\end{Highlighting}
\end{Shaded}

\begin{verbatim}
## [1] 1
\end{verbatim}

\begin{Shaded}
\begin{Highlighting}[]
\NormalTok{A}\OperatorTok{+}\DecValTok{6}
\end{Highlighting}
\end{Shaded}

\begin{verbatim}
## [1] 7
\end{verbatim}

Al correr una linea con el nombre del objeto, la consola del programa nos muestra su contenido. Entre Corchetes Observamos el número de orden del elemento en cuestión

El operador \textbf{\texttt{=}} es \textbf{equivalente} a \textbf{\texttt{\textless{}-}}, pero en la práctica no se utiliza para la definición de objetos.

\begin{Shaded}
\begin{Highlighting}[]
\NormalTok{B =}\StringTok{ }\DecValTok{2}
\NormalTok{B}
\end{Highlighting}
\end{Shaded}

\begin{verbatim}
## [1] 2
\end{verbatim}

\textbf{\texttt{\textless{}-}} es un operador \textbf{Unidireccional}, es decir que:\\
\texttt{A\ \textless{}-\ B} implica que \textbf{A} va tomar como valor el contenido del objeto \textbf{B}, y no al revés.

\begin{Shaded}
\begin{Highlighting}[]
\NormalTok{A <-}\StringTok{ }\NormalTok{B}
\NormalTok{A   }\CommentTok{#Ahora A toma el valor de B, y B continua conservando el mismo valor}
\end{Highlighting}
\end{Shaded}

\begin{verbatim}
## [1] 2
\end{verbatim}

\begin{Shaded}
\begin{Highlighting}[]
\NormalTok{B}
\end{Highlighting}
\end{Shaded}

\begin{verbatim}
## [1] 2
\end{verbatim}

\hypertarget{r-base}{%
\subsection{R base}\label{r-base}}

Con \emph{R base} nos referimos a los comandos básicos que vienen incorporados en el R, sin necesidad de cargar librerías.

\hypertarget{operadores-logicos}{%
\subsubsection{Operadores lógicos:}\label{operadores-logicos}}

\begin{itemize}
\tightlist
\item
  \(>\)
\item
  \(>=\)
\item
  \(<\)
\item
  \(<=\)
\item
  \(==\)
\item
  \(!=\)
\end{itemize}

\begin{Shaded}
\begin{Highlighting}[]
\CommentTok{#Redefinimos los valores A y B}
\NormalTok{A <-}\StringTok{  }\DecValTok{10}
\NormalTok{B  <-}\StringTok{  }\DecValTok{20}
\CommentTok{#Realizamos comparaciones lógicas}

\NormalTok{A }\OperatorTok{>}\StringTok{  }\NormalTok{B}
\end{Highlighting}
\end{Shaded}

\begin{verbatim}
## [1] FALSE
\end{verbatim}

\begin{Shaded}
\begin{Highlighting}[]
\NormalTok{A }\OperatorTok{>=}\StringTok{ }\NormalTok{B}
\end{Highlighting}
\end{Shaded}

\begin{verbatim}
## [1] FALSE
\end{verbatim}

\begin{Shaded}
\begin{Highlighting}[]
\NormalTok{A }\OperatorTok{<}\StringTok{  }\NormalTok{B}
\end{Highlighting}
\end{Shaded}

\begin{verbatim}
## [1] TRUE
\end{verbatim}

\begin{Shaded}
\begin{Highlighting}[]
\NormalTok{A }\OperatorTok{<=}\StringTok{ }\NormalTok{B}
\end{Highlighting}
\end{Shaded}

\begin{verbatim}
## [1] TRUE
\end{verbatim}

\begin{Shaded}
\begin{Highlighting}[]
\NormalTok{A }\OperatorTok{==}\StringTok{ }\NormalTok{B}
\end{Highlighting}
\end{Shaded}

\begin{verbatim}
## [1] FALSE
\end{verbatim}

\begin{Shaded}
\begin{Highlighting}[]
\NormalTok{A }\OperatorTok{!=}\StringTok{ }\NormalTok{B}
\end{Highlighting}
\end{Shaded}

\begin{verbatim}
## [1] TRUE
\end{verbatim}

\begin{Shaded}
\begin{Highlighting}[]
\NormalTok{C <-}\StringTok{ }\NormalTok{A }\OperatorTok{!=}\StringTok{ }\NormalTok{B}
\NormalTok{C}
\end{Highlighting}
\end{Shaded}

\begin{verbatim}
## [1] TRUE
\end{verbatim}

Como muestra el último ejemplo, el resultado de una operación lógica puede almacenarse como el valor de un objeto.

\hypertarget{operadores-aritmeticos}{%
\subsubsection{Operadores aritméticos:}\label{operadores-aritmeticos}}

\begin{Shaded}
\begin{Highlighting}[]
\CommentTok{#suma}
\NormalTok{A <-}\StringTok{ }\DecValTok{5}\OperatorTok{+}\DecValTok{6}
\NormalTok{A}
\end{Highlighting}
\end{Shaded}

\begin{verbatim}
## [1] 11
\end{verbatim}

\begin{Shaded}
\begin{Highlighting}[]
\CommentTok{#Resta}
\NormalTok{B <-}\StringTok{ }\DecValTok{6-8}
\NormalTok{B}
\end{Highlighting}
\end{Shaded}

\begin{verbatim}
## [1] -2
\end{verbatim}

\begin{Shaded}
\begin{Highlighting}[]
\CommentTok{#cociente}
\NormalTok{C <-}\StringTok{ }\DecValTok{6}\OperatorTok{/}\FloatTok{2.5}
\NormalTok{C}
\end{Highlighting}
\end{Shaded}

\begin{verbatim}
## [1] 2.4
\end{verbatim}

\begin{Shaded}
\begin{Highlighting}[]
\CommentTok{#multiplicacion}
\NormalTok{D <-}\StringTok{ }\DecValTok{6}\OperatorTok{*}\FloatTok{2.5}
\NormalTok{D}
\end{Highlighting}
\end{Shaded}

\begin{verbatim}
## [1] 15
\end{verbatim}

\hypertarget{funciones}{%
\subsubsection{Funciones:}\label{funciones}}

Las funciones son series de procedimientos estandarizados, que toman como imput determinados argumentos a fijar por el usuario, y devuelven un resultado acorde a la aplicación de dichos procedimientos. Su lógica de funcionamiento es:\\
\texttt{funcion(argumento1\ =\ arg1,\ argumento2\ =\ arg2)}

A lo largo del curso iremos viendo numerosas funciones, según lo requieran los distintos ejercicios. Sin embargo, veamos ahora algunos ejemplos para comprender su funcionamiento:

\begin{itemize}
\tightlist
\item
  paste() : concatena una serie de caracteres, indicando por última instancia como separar a cada uno de ellos\\
\item
  paste0(): concatena una serie de caracteres sin separar
\item
  sum(): suma de todos los elementos de un vector\\
\item
  mean() promedio aritmético de todos los elementos de un vector
\end{itemize}

\begin{Shaded}
\begin{Highlighting}[]
\KeywordTok{paste}\NormalTok{(}\StringTok{"Pega"}\NormalTok{,}\StringTok{"estas"}\NormalTok{,}\DecValTok{4}\NormalTok{,}\StringTok{"palabras"}\NormalTok{, }\DataTypeTok{sep =} \StringTok{" "}\NormalTok{)}
\end{Highlighting}
\end{Shaded}

\begin{verbatim}
## [1] "Pega estas 4 palabras"
\end{verbatim}

\begin{Shaded}
\begin{Highlighting}[]
\CommentTok{#Puedo concatenar caracteres almacenados en objetos}
\KeywordTok{paste}\NormalTok{(A,B,C,}\DataTypeTok{sep =} \StringTok{"**"}\NormalTok{)}
\end{Highlighting}
\end{Shaded}

\begin{verbatim}
## [1] "11**-2**2.4"
\end{verbatim}

\begin{Shaded}
\begin{Highlighting}[]
\CommentTok{# Paste0 pega los caracteres sin separador}
\KeywordTok{paste0}\NormalTok{(A,B,C)}
\end{Highlighting}
\end{Shaded}

\begin{verbatim}
## [1] "11-22.4"
\end{verbatim}

\begin{Shaded}
\begin{Highlighting}[]
\DecValTok{1}\OperatorTok{:}\DecValTok{5}
\end{Highlighting}
\end{Shaded}

\begin{verbatim}
## [1] 1 2 3 4 5
\end{verbatim}

\begin{Shaded}
\begin{Highlighting}[]
\KeywordTok{sum}\NormalTok{(}\DecValTok{1}\OperatorTok{:}\DecValTok{5}\NormalTok{)}
\end{Highlighting}
\end{Shaded}

\begin{verbatim}
## [1] 15
\end{verbatim}

\begin{Shaded}
\begin{Highlighting}[]
\KeywordTok{mean}\NormalTok{(}\DecValTok{1}\OperatorTok{:}\DecValTok{5}\NormalTok{,}\DataTypeTok{na.rm =} \OtherTok{TRUE}\NormalTok{)}
\end{Highlighting}
\end{Shaded}

\begin{verbatim}
## [1] 3
\end{verbatim}

\hypertarget{caracteres-especiales}{%
\subsubsection{Caracteres especiales}\label{caracteres-especiales}}

\begin{itemize}
\tightlist
\item
  R es sensible a mayúsculas y minúsculas, tanto para los nombres de las variables, como para las funciones y parámetros.
\item
  Los \textbf{espacios en blanco} y los \textbf{carriage return} (\emph{enter}) no son considerados por el lenguaje. Los podemos aprovechar para emprolijar el código y que la lectura sea más simple\footnote{veremos que existen ciertas excepciones con algunos paquetes más adelante.}.
\end{itemize}

\begin{itemize}
\item
  El \textbf{numeral} \texttt{\#} se utiliza para hacer comentarios. Todo lo que se escribe después del \# no es interpretado por R. Se debe utilizar un \# por cada línea de código que se desea anular
\item
  Los \textbf{corchetes} \texttt{{[}{]}} se utilizan para acceder a un objeto:

  \begin{itemize}
  \tightlist
  \item
    en un vector{[}n° orden{]}
  \item
    en una tabla{[}fila, columna{]}
  \item
    en una lista{[}n° elemento{]}
  \end{itemize}
\item
  el signo \textbf{\$} también es un método de acceso. Particularmente, en los dataframes, nos permitira acceder a una determinada columna de una tabla
\item
  Los \textbf{paréntesis}\texttt{()} se utilizan en las funciones para definir los parámetros.
\item
  Las \textbf{comas} \texttt{,} se utilizan para separar los parametros al interior de una función.
\end{itemize}

\hypertarget{objetos}{%
\subsection{Objetos:}\label{objetos}}

Existen un gran cantidad de objetos distintos en R, en lo que resepcta al curso trabajaremos principalmente con 3 de ellos:

\begin{itemize}
\tightlist
\item
  Valores
\item
  Vectores
\item
  Data Frames
\item
  Listas
\end{itemize}

\hypertarget{valores}{%
\subsubsection{Valores}\label{valores}}

Los valores y vectores pueden ser a su vez de distintas \emph{clases}:

\textbf{Numeric}

\begin{Shaded}
\begin{Highlighting}[]
\NormalTok{A <-}\StringTok{  }\DecValTok{1}
\KeywordTok{class}\NormalTok{(A)}
\end{Highlighting}
\end{Shaded}

\begin{verbatim}
## [1] "numeric"
\end{verbatim}

\textbf{Character}

\begin{Shaded}
\begin{Highlighting}[]
\NormalTok{A <-}\StringTok{  }\KeywordTok{paste}\NormalTok{(}\StringTok{'Soy'}\NormalTok{, }\StringTok{'una'}\NormalTok{, }\StringTok{'concatenación', '}\NormalTok{de}\StringTok{', '}\NormalTok{caracteres}\StringTok{', sep = " ")}
\StringTok{A}
\end{Highlighting}
\end{Shaded}

\begin{verbatim}
## [1] "Soy una concatenación de caracteres"
\end{verbatim}

\begin{Shaded}
\begin{Highlighting}[]
\KeywordTok{class}\NormalTok{(A)}
\end{Highlighting}
\end{Shaded}

\begin{verbatim}
## [1] "character"
\end{verbatim}

\textbf{Factor}

\begin{Shaded}
\begin{Highlighting}[]
\NormalTok{A <-}\StringTok{ }\KeywordTok{factor}\NormalTok{(}\StringTok{"Soy un factor, con niveles fijos"}\NormalTok{)}
\KeywordTok{class}\NormalTok{(A)}
\end{Highlighting}
\end{Shaded}

\begin{verbatim}
## [1] "factor"
\end{verbatim}

La diferencia entre un \emph{character} y un \emph{factor} es que el último tiene solo algunos valores permitidos (levels), con un orden interno predefinido (el cual ,por ejemplo, se respetará a la hora de realizar un gráfico)

\hypertarget{vectores}{%
\subsubsection{Vectores}\label{vectores}}

Para crear un \textbf{vector} utilizamos el comando \texttt{c()}, de combinar.

\begin{Shaded}
\begin{Highlighting}[]
\NormalTok{C <-}\StringTok{ }\KeywordTok{c}\NormalTok{(}\DecValTok{1}\NormalTok{, }\DecValTok{3}\NormalTok{, }\DecValTok{4}\NormalTok{)}
\NormalTok{C}
\end{Highlighting}
\end{Shaded}

\begin{verbatim}
## [1] 1 3 4
\end{verbatim}

sumarle 2 a cada elemento del \textbf{vector} anterior

\begin{Shaded}
\begin{Highlighting}[]
\NormalTok{C <-}\StringTok{ }\NormalTok{C }\OperatorTok{+}\StringTok{ }\DecValTok{2}
\NormalTok{C}
\end{Highlighting}
\end{Shaded}

\begin{verbatim}
## [1] 3 5 6
\end{verbatim}

sumarle 1 al primer elemento, 2 al segundo, y 3 al tercer elemento del \textbf{vector} anterior

\begin{Shaded}
\begin{Highlighting}[]
\NormalTok{D <-}\StringTok{ }\NormalTok{C }\OperatorTok{+}\StringTok{ }\DecValTok{1}\OperatorTok{:}\DecValTok{3} \CommentTok{#esto es equivalente a hacer 3+1, 5+2, 6+9 }
\NormalTok{D}
\end{Highlighting}
\end{Shaded}

\begin{verbatim}
## [1] 4 7 9
\end{verbatim}

\texttt{1:3} significa que queremos todos los números enteros desde 1 hasta 3.

crear un \textbf{vector} que contenga las palabras: ``Carlos'',``Federico'',``Pedro''

\begin{Shaded}
\begin{Highlighting}[]
\NormalTok{E <-}\StringTok{ }\KeywordTok{c}\NormalTok{(}\StringTok{"Carlos"}\NormalTok{,}\StringTok{"Federico"}\NormalTok{,}\StringTok{"Pedro"}\NormalTok{)}
\NormalTok{E}
\end{Highlighting}
\end{Shaded}

\begin{verbatim}
## [1] "Carlos"   "Federico" "Pedro"
\end{verbatim}

para acceder a algún elemento del vector, podemos buscarlo por su número de orden, entre \texttt{{[}\ {]}}

\begin{Shaded}
\begin{Highlighting}[]
\NormalTok{ E[}\DecValTok{2}\NormalTok{]}
\end{Highlighting}
\end{Shaded}

\begin{verbatim}
## [1] "Federico"
\end{verbatim}

Si nos interesa almacenar dicho valor, al buscarlo lo asignamos a un nuevo objeto, dandole el nombre que deseemos

\begin{Shaded}
\begin{Highlighting}[]
\NormalTok{elemento2 <-}\StringTok{  }\NormalTok{E[}\DecValTok{2}\NormalTok{]}
\end{Highlighting}
\end{Shaded}

\begin{Shaded}
\begin{Highlighting}[]
\NormalTok{elemento2}
\end{Highlighting}
\end{Shaded}

\begin{verbatim}
## [1] "Federico"
\end{verbatim}

para \textbf{borrar} un objeto del ambiente de trabajo, utilizamos el comando \emph{\texttt{rm()}}

\begin{Shaded}
\begin{Highlighting}[]
\KeywordTok{rm}\NormalTok{(elemento2)}
\NormalTok{elemento2}
\end{Highlighting}
\end{Shaded}

\begin{verbatim}
## Error in eval(expr, envir, enclos): object 'elemento2' not found
\end{verbatim}

También podemos cambiar el texto del segundo elemento de E, por el texto ``Pablo''

\begin{Shaded}
\begin{Highlighting}[]
\NormalTok{E[}\DecValTok{2}\NormalTok{] <-}\StringTok{ "Pablo"}
\NormalTok{E}
\end{Highlighting}
\end{Shaded}

\begin{verbatim}
## [1] "Carlos" "Pablo"  "Pedro"
\end{verbatim}

\hypertarget{data-frames}{%
\subsection{Data Frames}\label{data-frames}}

Un Data Frame es una tabla de datos, donde cada columna representa una variable, y cada fila una observación.

Este objeto suele ser central en el proceso de trabajo, y suele ser la forma en que se cargan datos externos para trabajar en el ambiente de R, y en que se exportan los resultados de nuestros trabajo.

También Se puede crear como la combinación de N vectores de igual tamaño. Por ejemplo, tomamos algunos valores del \href{http://www.indec.gob.ar/bajarCuadroEstadistico.asp?idc=4020B33440609462654542BD0BC320F1523DA0DC52C396201DB4DD5861FFEDC9AD1436681AC84179}{Indice de salarios}

\begin{Shaded}
\begin{Highlighting}[]
\NormalTok{INDICE  <-}\StringTok{ }\KeywordTok{c}\NormalTok{(}\DecValTok{100}\NormalTok{,   }\DecValTok{100}\NormalTok{,   }\DecValTok{100}\NormalTok{,}
             \FloatTok{101.8}\NormalTok{, }\FloatTok{101.2}\NormalTok{, }\FloatTok{100.73}\NormalTok{,}
             \FloatTok{102.9}\NormalTok{, }\FloatTok{102.4}\NormalTok{, }\FloatTok{103.2}\NormalTok{)}

\NormalTok{FECHA  <-}\StringTok{  }\KeywordTok{c}\NormalTok{(}\StringTok{"Oct-16"}\NormalTok{, }\StringTok{"Oct-16"}\NormalTok{, }\StringTok{"Oct-16"}\NormalTok{,}
             \StringTok{"Nov-16"}\NormalTok{, }\StringTok{"Nov-16"}\NormalTok{, }\StringTok{"Nov-16"}\NormalTok{,}
             \StringTok{"Dic-16"}\NormalTok{, }\StringTok{"Dic-16"}\NormalTok{, }\StringTok{"Dic-16"}\NormalTok{)}


\NormalTok{GRUPO  <-}\StringTok{  }\KeywordTok{c}\NormalTok{(}\StringTok{"Privado_Registrado"}\NormalTok{,}\StringTok{"Público","}\NormalTok{Privado_No_Registrado}\StringTok{",}
\StringTok{             "}\NormalTok{Privado_Registrado}\StringTok{","}\NormalTok{Público",}\StringTok{"Privado_No_Registrado"}\NormalTok{,}
             \StringTok{"Privado_Registrado"}\NormalTok{,}\StringTok{"Público","}\NormalTok{Privado_No_Registrado}\StringTok{")}
\StringTok{             }

\StringTok{Datos <- data.frame(INDICE, FECHA, GRUPO)}
\StringTok{Datos}
\end{Highlighting}
\end{Shaded}

\begin{verbatim}
##   INDICE  FECHA                 GRUPO
## 1 100.00 Oct-16    Privado_Registrado
## 2 100.00 Oct-16               Público
## 3 100.00 Oct-16 Privado_No_Registrado
## 4 101.80 Nov-16    Privado_Registrado
## 5 101.20 Nov-16               Público
## 6 100.73 Nov-16 Privado_No_Registrado
## 7 102.90 Dic-16    Privado_Registrado
## 8 102.40 Dic-16               Público
## 9 103.20 Dic-16 Privado_No_Registrado
\end{verbatim}

Tal como en un \textbf{vector} se ubica a los elementos mediante \texttt{{[}\ {]}}, en un \textbf{dataframe} se obtienen sus elementos de la forma \textbf{\texttt{{[}fila,\ columna{]}}}.

Otra opción es especificar la columna, mediante el operador \textbf{\texttt{\$}}, y luego seleccionar dentro de esa columna el registro deseado mediante el número de orden.

\begin{Shaded}
\begin{Highlighting}[]
\NormalTok{Datos}\OperatorTok{$}\NormalTok{FECHA}
\end{Highlighting}
\end{Shaded}

\begin{verbatim}
## [1] Oct-16 Oct-16 Oct-16 Nov-16 Nov-16 Nov-16 Dic-16 Dic-16 Dic-16
## Levels: Dic-16 Nov-16 Oct-16
\end{verbatim}

\begin{Shaded}
\begin{Highlighting}[]
\NormalTok{Datos[}\DecValTok{3}\NormalTok{,}\DecValTok{2}\NormalTok{]}
\end{Highlighting}
\end{Shaded}

\begin{verbatim}
## [1] Oct-16
## Levels: Dic-16 Nov-16 Oct-16
\end{verbatim}

\begin{Shaded}
\begin{Highlighting}[]
\NormalTok{Datos}\OperatorTok{$}\NormalTok{FECHA[}\DecValTok{3}\NormalTok{]}
\end{Highlighting}
\end{Shaded}

\begin{verbatim}
## [1] Oct-16
## Levels: Dic-16 Nov-16 Oct-16
\end{verbatim}

¿que pasa si hacemos \texttt{Datos\$FECHA{[}3,2{]}} ?

\begin{Shaded}
\begin{Highlighting}[]
\NormalTok{Datos}\OperatorTok{$}\NormalTok{FECHA[}\DecValTok{3}\NormalTok{,}\DecValTok{2}\NormalTok{]}
\end{Highlighting}
\end{Shaded}

\begin{verbatim}
## Error in `[.default`(Datos$FECHA, 3, 2): incorrect number of dimensions
\end{verbatim}

Nótese que el último comando tiene un número incorrecto de dimensiones, porque estamos refiriendonos 2 veces a la columna FECHA.

Acorde a lo visto anteriormente, el acceso a los \textbf{dataframes} mediante \texttt{{[}\ {]}}, puede utilizarse para realizar filtros sobre la base, especificando una condición para las filas. Por ejemplo, puedo utilizar los \texttt{{[}\ {]}} para conservar del \textbf{dataframe} \texttt{Datos} unicamente los registros con fecha de Diciembre 2016:

\begin{Shaded}
\begin{Highlighting}[]
\NormalTok{Datos[Datos}\OperatorTok{$}\NormalTok{FECHA}\OperatorTok{==}\StringTok{"Dic-16"}\NormalTok{,]}
\end{Highlighting}
\end{Shaded}

\begin{verbatim}
##   INDICE  FECHA                 GRUPO
## 7  102.9 Dic-16    Privado_Registrado
## 8  102.4 Dic-16               Público
## 9  103.2 Dic-16 Privado_No_Registrado
\end{verbatim}

La lógica del paso anterior sería: Accedo al dataframe \texttt{Datos}, pidiendo únicamente conservar las filas (por eso la condición se ubica a la \emph{izquierda} de la \texttt{,}) que cumplan el requisito de pertenecer a la categoría \textbf{``Dic-16''} de la variable \textbf{FECHA}.\\
Aún más, podría aplicar el filtro y al mismo tiempo identificar una variable de interés para luego realizar un cálculo sobre aquella. Por ejemplo, podría calcular la media de los indices en el mes de Diciembre.

\begin{Shaded}
\begin{Highlighting}[]
\CommentTok{###Por separado}
\NormalTok{Indices_Dic <-}\StringTok{ }\NormalTok{Datos}\OperatorTok{$}\NormalTok{INDICE[Datos}\OperatorTok{$}\NormalTok{FECHA}\OperatorTok{==}\StringTok{"Dic-16"}\NormalTok{]}
\NormalTok{Indices_Dic}
\end{Highlighting}
\end{Shaded}

\begin{verbatim}
## [1] 102.9 102.4 103.2
\end{verbatim}

\begin{Shaded}
\begin{Highlighting}[]
\KeywordTok{mean}\NormalTok{(Indices_Dic)}
\end{Highlighting}
\end{Shaded}

\begin{verbatim}
## [1] 102.8333
\end{verbatim}

\begin{Shaded}
\begin{Highlighting}[]
\CommentTok{### Todo junto}
\KeywordTok{mean}\NormalTok{(Datos}\OperatorTok{$}\NormalTok{INDICE[Datos}\OperatorTok{$}\NormalTok{FECHA}\OperatorTok{==}\StringTok{"Dic-16"}\NormalTok{])}
\end{Highlighting}
\end{Shaded}

\begin{verbatim}
## [1] 102.8333
\end{verbatim}

La lógica de esta sintaxis sería: ``Me quedó con la variable \textbf{INDICE}, cuando la variable FECHA sea igual a \textbf{"Dic-16"}, luego calculo la media de dichos valores''

\hypertarget{listas}{%
\subsection{Listas}\label{listas}}

Contienen una concatenación de objetos de cualquier tipo. Así como un vector contiene valores, un dataframe contiene vectores, una lista puede contener dataframes, pero también vectores, o valores, y \emph{todo ello a la vez}

\begin{Shaded}
\begin{Highlighting}[]
\NormalTok{superlista <-}\StringTok{ }\KeywordTok{list}\NormalTok{(A,B,C,D,E,FECHA, }\DataTypeTok{DF =}\NormalTok{ Datos, INDICE, GRUPO)}
\NormalTok{superlista}
\end{Highlighting}
\end{Shaded}

\begin{verbatim}
## [[1]]
## [1] Soy un factor, con niveles fijos
## Levels: Soy un factor, con niveles fijos
## 
## [[2]]
## [1] -2
## 
## [[3]]
## [1] 3 5 6
## 
## [[4]]
## [1] 4 7 9
## 
## [[5]]
## [1] "Carlos" "Pablo"  "Pedro" 
## 
## [[6]]
## [1] "Oct-16" "Oct-16" "Oct-16" "Nov-16" "Nov-16" "Nov-16" "Dic-16" "Dic-16"
## [9] "Dic-16"
## 
## $DF
##   INDICE  FECHA                 GRUPO
## 1 100.00 Oct-16    Privado_Registrado
## 2 100.00 Oct-16               Público
## 3 100.00 Oct-16 Privado_No_Registrado
## 4 101.80 Nov-16    Privado_Registrado
## 5 101.20 Nov-16               Público
## 6 100.73 Nov-16 Privado_No_Registrado
## 7 102.90 Dic-16    Privado_Registrado
## 8 102.40 Dic-16               Público
## 9 103.20 Dic-16 Privado_No_Registrado
## 
## [[8]]
## [1] 100.00 100.00 100.00 101.80 101.20 100.73 102.90 102.40 103.20
## 
## [[9]]
## [1] "Privado_Registrado"    "Público"               "Privado_No_Registrado"
## [4] "Privado_Registrado"    "Público"               "Privado_No_Registrado"
## [7] "Privado_Registrado"    "Público"               "Privado_No_Registrado"
\end{verbatim}

Para acceder un elemento de una lista, podemos utilizar el operador \textbf{\texttt{\$}}, que se puede usar a su vez de forma iterativa

\begin{Shaded}
\begin{Highlighting}[]
\NormalTok{superlista}\OperatorTok{$}\NormalTok{DF}\OperatorTok{$}\NormalTok{FECHA[}\DecValTok{2}\NormalTok{]}
\end{Highlighting}
\end{Shaded}

\begin{verbatim}
## [1] Oct-16
## Levels: Dic-16 Nov-16 Oct-16
\end{verbatim}

\hypertarget{ambientes-de-trabajo}{%
\subsection{Ambientes de trabajo}\label{ambientes-de-trabajo}}

Hay algunas cosas que tenemos que tener en cuenta respecto del orden del ambiente en el que trabajamos:

\begin{itemize}
\tightlist
\item
  Working Directory: El directorio de trabajo, pueden ver el suyo con \texttt{getwd()}, es \emph{hacia donde apunta el código}, por ejemplo, si quieren leer un archivo, la ruta del archivo tiene que estar explicitada como el recorrido desde el Working Directory.
\item
  Environment: Esto engloba tanto la información que tenemos cargada en \emph{Data} y \emph{Values}, como las librerías que tenemos cargadas mientras trabajamos.
\end{itemize}

Es importante que mantengamos bien delimitadas estas cosas entre diferentes trabajos, sino:

\begin{enumerate}
\def\labelenumi{\arabic{enumi}.}
\tightlist
\item
  El directorio queda referido a un lugar específico en nuestra computadora.
\end{enumerate}

\begin{itemize}
\tightlist
\item
  Si se lo compartimos a otro \textbf{se rompe}
\item
  Si cambiamos de computadora \textbf{se rompe}
\item
  Si lo cambiamos de lugar \textbf{se rompe}
\item
  Si primero abrimos otro script \textbf{se rompe}
\end{itemize}

\begin{enumerate}
\def\labelenumi{\arabic{enumi}.}
\setcounter{enumi}{1}
\tightlist
\item
  Tenemos mezclados resultados de diferentes trabajos:
\end{enumerate}

\begin{itemize}
\tightlist
\item
  Nunca sabemos si esa variable/tabla/lista se creo en ese script y no otro
\item
  Perdemos espacio de la memoria
\item
  No estamos seguros de que el script cargue todas las librerías que necesita
\end{itemize}

Rstudio tiene una herramienta muy útil de trabajo que son los \textbf{proyectos}. Estos permiten mantener un ambiente de trabajo delimitado por cada uno de nuestros trabajos. Es decir:

\begin{itemize}
\tightlist
\item
  El directorio de trabajo se refiere a donde esta ubicado el archivo .Rproj
\item
  El Environment es específico de nuestro proyecto.
\end{itemize}

Un proyecto no es un sólo script, sino toda una carpeta de trabajo.

\begin{figure}
\centering
\includegraphics{img/Rproject.png}
\caption{logo Rpoject}
\end{figure}

Para crearlo, vamos al logo de nuevo projecto (Arriba a la izquierda de la panatalla), y elegimos la carpeta de trabajo.

\hypertarget{tipos-de-archivos-de-r}{%
\subsection{Tipos de archivos de R}\label{tipos-de-archivos-de-r}}

\begin{itemize}
\tightlist
\item
  \textbf{Script}: Es un archivo de texto plano, donde podemos poner el código que utilizamos para preservarlo
\item
  \textbf{Rnotebook}: También sirve para guardar el código, pero a diferencia de los scripts, se puede compilar, e intercalar código con resultados (este archivo es un rnotebook)
\item
  \textbf{Rproject}: Es un archivo que define la metadata del proyecto
\item
  \textbf{RDS y Rdata}: Dos formatos de archivos propios de R para guardar datos.
\end{itemize}

\hypertarget{practica-guiada}{%
\section{Práctica Guiada}\label{practica-guiada}}

\hypertarget{instalacion-de-paquetes-complementarios-al-r-base}{%
\subsection{Instalación de paquetes complementarios al R Base}\label{instalacion-de-paquetes-complementarios-al-r-base}}

Hasta aquí hemos visto múltiples funciones que están contenidas dentro del lenguaje básico de R. Ahora bien, al tratarse de un software libre, los usuarios de R con más experiencia contribuyen sistemáticamente a expandir este lenguaje mediante la creación y actualización de \textbf{paquetes} complementarios. Lógicamente, los mismos no están incluidos en la instalación inicial del programa, pero podemos descargarlos e instalarlos al mismo tiempo con el siguiente comando:

\begin{verbatim}
install.packages("nombre_del_paquete") 
\end{verbatim}

Resulta recomendable \textbf{ejecutar este comando desde la consola} ya que solo necesitaremos correrlo una vez en nuestra computadora. Al ejecutar el mismo, se descargarán de la pagina de \href{www.cran.r-project.org}{CRAN} los archivos correspondientes al paquete hacia el directorio en donde hayamos instalado el programa. Típicamente los archivos se encontrarán en \textbf{\texttt{C:\textbackslash{}Program\ Files\textbackslash{}R\textbackslash{}R-3.5.0\textbackslash{}library\textbackslash{}}}, siempre con la versión del programa correspondiente.\\
Una vez instalado el paquete, cada vez que abramos una nueva sesión de R y querramos utilizar el mismo debemos \textbf{cargarlo al ambiente de trabajo} mediante la siguiente función:

\begin{verbatim}
library(nombre_del_paquete)
\end{verbatim}

Nótese que al cargar/activar el paquete no son necesarias las comillas.

\hypertarget{lectura-y-escritura-de-archivos}{%
\subsection{Lectura y escritura de archivos}\label{lectura-y-escritura-de-archivos}}

\hypertarget{csv-y-.txt}{%
\subsubsection{.csv y .txt}\label{csv-y-.txt}}

Hay \textbf{muchas} funciones para leer archivos de tipo \emph{.txt} y \emph{.csv}. La mayoría sólo cambia los parámetros que vienen por default.

Es importante tener en cuenta que una base de datos que proviene de archivos \emph{.txt}, o \emph{.csv} puede presentar diferencias en cuanto a los siguientes parametros:

\begin{itemize}
\tightlist
\item
  encabezado
\item
  delimitador (\texttt{,}, tab, \texttt{;})
\item
  separador decimal
\end{itemize}

\begin{verbatim}
dataframe <- read.delim(file, header = TRUE, sep = "\t", quote = "\"", dec = ".", fill = TRUE, comment.char = "", ...) 
\end{verbatim}

Ejemplo. Levantar la base de \href{https://data.buenosaires.gob.ar/dataset/sueldo-funcionarios}{sueldos de funcionarios}

En el parametro \texttt{file} tengo que especificar el nombre completo del archivo, incluyendo el directorio donde se encuentra. Lo más sencillo es abrir comillas, apretar \texttt{Tab} y se despliega el menú de las cosas que tenemos en el directorio de trabajo. Si queremos movernos hacia arriba, agregamos \texttt{../}

\begin{Shaded}
\begin{Highlighting}[]
\NormalTok{sueldos_funcionarios <-}\StringTok{ }\KeywordTok{read.table}\NormalTok{(}\DataTypeTok{file =} \StringTok{'../fuentes/sueldo_funcionarios_2019.csv'}\NormalTok{,}\DataTypeTok{sep=}\StringTok{","}\NormalTok{, }\DataTypeTok{header =} \OtherTok{TRUE}\NormalTok{)}
\NormalTok{sueldos_funcionarios[}\DecValTok{1}\OperatorTok{:}\DecValTok{10}\NormalTok{,]}
\end{Highlighting}
\end{Shaded}

\begin{verbatim}
##             cuil anio mes funcionario_apellido funcionario_nombre
## 1  20-17692128-6 2019   1    RODRIGUEZ LARRETA    HORACIO ANTONIO
## 2  20-17735449-0 2019   1             SANTILLI        DIEGO CESAR
## 3  27-24483014-0 2019   1                ACUÑA      MARIA SOLEDAD
## 4  20-13872301-2 2019   1             ASTARLOA      GABRIEL MARIA
## 5  20-25641207-2 2019   1             AVOGADRO       ENRIQUE LUIS
## 6  27-13221055-7 2019   1            BOU PEREZ          ANA MARIA
## 7  27-13092400-5 2019   1                FREDA     MONICA BEATRIZ
## 8  20-17110752-1 2019   1         MACCHIAVELLI    EDUARDO ALBERTO
## 9  20-22293873-3 2019   1               MIGUEL       FELIPE OSCAR
## 10 20-14699669-9 2019   1               MOCCIA             FRANCO
##                                         repartición asignacion_por_cargo_i
## 1                                  Jefe de Gobierno               197745.8
## 2                          Vicejefatura de Gobierno               197745.8
## 3              Ministerio de Educación e Innovación               224516.6
## 4  Procuración General de la Ciudad de Buenos Aires               224516.6
## 5                             Ministerio de Cultura               224516.6
## 6                               Ministerio de Salud               224516.6
## 7  Sindicatura General de la Ciudad de Buenos Aires               224516.6
## 8          Ministerio de Ambiente y Espacio Público               224516.6
## 9                 Jefatura de Gabinete de Ministros               224516.6
## 10     Ministerio de Desarrollo Urbano y Transporte               224516.6
##    aguinaldo_ii total_salario_bruto_i_._ii observaciones
## 1             0                   197745.8              
## 2             0                   197745.8              
## 3             0                   224516.6              
## 4             0                   224516.6              
## 5             0                   224516.6              
## 6             0                   224516.6              
## 7             0                   224516.6              
## 8             0                   224516.6              
## 9             0                   224516.6              
## 10            0                   224516.6
\end{verbatim}

Como puede observarse aquí, las bases individuales de la EPH cuentan con más de 58.000 registros y 177 variables.
Al trabajar con bases de microdatos, resulta conveniente contar con algunos comandos para tener una mirada rápida de la base, antes de comenzar a realizar los procesamientos que deseemos.

Veamos algunos de ellos:

\begin{Shaded}
\begin{Highlighting}[]
\CommentTok{#View(individual_t117)}
\KeywordTok{names}\NormalTok{(sueldos_funcionarios)}
\end{Highlighting}
\end{Shaded}

\begin{verbatim}
##  [1] "cuil"                       "anio"                      
##  [3] "mes"                        "funcionario_apellido"      
##  [5] "funcionario_nombre"         "repartición"               
##  [7] "asignacion_por_cargo_i"     "aguinaldo_ii"              
##  [9] "total_salario_bruto_i_._ii" "observaciones"
\end{verbatim}

\begin{Shaded}
\begin{Highlighting}[]
\KeywordTok{summary}\NormalTok{(sueldos_funcionarios)}
\end{Highlighting}
\end{Shaded}

\begin{verbatim}
##             cuil         anio           mes       funcionario_apellido
##  20-13872301-2: 3   Min.   :2019   Min.   :1.00   ACUÑA     : 3       
##  20-14699669-9: 3   1st Qu.:2019   1st Qu.:2.00   ASTARLOA  : 3       
##  20-16891528-5: 3   Median :2019   Median :3.00   AVELLANEDA: 3       
##  20-16891539-0: 3   Mean   :2019   Mean   :3.34   AVOGADRO  : 3       
##  20-17110752-1: 3   3rd Qu.:2019   3rd Qu.:5.00   BENEGAS   : 3       
##  20-17692128-6: 3   Max.   :2019   Max.   :6.00   BOU PEREZ : 3       
##  (Other)      :76                                 (Other)   :76       
##         funcionario_nombre
##   ANA MARIA      : 3      
##   BRUNO GUIDO    : 3      
##   CHRISTIAN      : 3      
##   DIEGO CESAR    : 3      
##   DIEGO HERNAN   : 3      
##   EDUARDO ALBERTO: 3      
##  (Other)         :76      
##                                                          repartición
##  Consejo de los Derechos de Niñas, Niños y Adoles - Presidencia: 3  
##  Ente de Turismo Ley Nº 2627                                   : 3  
##  Jefatura de Gabinete de Ministros                             : 3  
##  Jefe de Gobierno                                              : 3  
##  Ministerio de Ambiente y Espacio Público                      : 3  
##  Ministerio de Cultura                                         : 3  
##  (Other)                                                       :76  
##  asignacion_por_cargo_i  aguinaldo_ii    total_salario_bruto_i_._ii
##  Min.   :197746         Min.   :     0   Min.   :197746            
##  1st Qu.:217520         1st Qu.:     0   1st Qu.:217805            
##  Median :226866         Median :     0   Median :226866            
##  Mean   :224718         Mean   : 14843   Mean   :239560            
##  3rd Qu.:231168         3rd Qu.:     0   3rd Qu.:248033            
##  Max.   :249662         Max.   :113433   Max.   :340300            
##                                                                    
##         observaciones
##                :93   
##  baja 28/2/2019: 1   
##                      
##                      
##                      
##                      
## 
\end{verbatim}

\begin{Shaded}
\begin{Highlighting}[]
\KeywordTok{head}\NormalTok{(sueldos_funcionarios)[,}\DecValTok{1}\OperatorTok{:}\DecValTok{5}\NormalTok{]}
\end{Highlighting}
\end{Shaded}

\begin{verbatim}
##            cuil anio mes funcionario_apellido funcionario_nombre
## 1 20-17692128-6 2019   1    RODRIGUEZ LARRETA    HORACIO ANTONIO
## 2 20-17735449-0 2019   1             SANTILLI        DIEGO CESAR
## 3 27-24483014-0 2019   1                ACUÑA      MARIA SOLEDAD
## 4 20-13872301-2 2019   1             ASTARLOA      GABRIEL MARIA
## 5 20-25641207-2 2019   1             AVOGADRO       ENRIQUE LUIS
## 6 27-13221055-7 2019   1            BOU PEREZ          ANA MARIA
\end{verbatim}

\hypertarget{excel}{%
\subsubsection{Excel}\label{excel}}

Para leer y escribir archivos excel debemos utilizar los comandos que vienen con la librería openxlsx

\begin{Shaded}
\begin{Highlighting}[]
\CommentTok{# install.packages("openxlsx") # por única vez}
\KeywordTok{library}\NormalTok{(openxlsx) }\CommentTok{#activamos la librería}

\CommentTok{#creamos una tabla cualquiera de prueba}
\NormalTok{x <-}\StringTok{ }\DecValTok{1}\OperatorTok{:}\DecValTok{10}
\NormalTok{y <-}\StringTok{ }\DecValTok{11}\OperatorTok{:}\DecValTok{20}
\NormalTok{tabla_de_R <-}\StringTok{ }\KeywordTok{data.frame}\NormalTok{(x,y)}

\CommentTok{# escribimos el archivo}
\KeywordTok{write.xlsx}\NormalTok{( }\DataTypeTok{x =}\NormalTok{ tabla_de_R, }\DataTypeTok{file =} \StringTok{"../resultados/archivo.xlsx"}\NormalTok{,}\DataTypeTok{row.names =} \OtherTok{FALSE}\NormalTok{)}
\CommentTok{#Donde lo guardó? Hay un directorio por default en caso de que no hayamos definido alguno.}

\CommentTok{#getwd()}

\CommentTok{#Si queremos exportar multiples dataframes a un Excel, debemos armar previamente una lista de ellos. Cada dataframe, se guardará en una pestaña de excel, cuyo nombre correspondera al que definamos para cada Dataframe a la hora de crear la lista.}
\NormalTok{Lista_a_exportar <-}\StringTok{ }\KeywordTok{list}\NormalTok{(}\StringTok{"sueldos funcionarios"}\NormalTok{ =}\StringTok{ }\NormalTok{sueldos_funcionarios,}
                         \StringTok{"Tabla Numeros"}\NormalTok{ =}\StringTok{ }\NormalTok{tabla_de_R)}

\KeywordTok{write.xlsx}\NormalTok{( }\DataTypeTok{x =}\NormalTok{ Lista_a_exportar, }\DataTypeTok{file =} \StringTok{"../resultados/archivo_2_hojas.xlsx"}\NormalTok{,}\DataTypeTok{row.names =} \OtherTok{FALSE}\NormalTok{)}

\CommentTok{#leemos el archivo especificando la ruta (o el directorio por default) y el nombre de la hoja que contiene los datos}
\NormalTok{Indices_Salario <-}\StringTok{ }\KeywordTok{read.xlsx}\NormalTok{(}\DataTypeTok{xlsxFile =} \StringTok{"../resultados/archivo_2_hojas.xlsx"}\NormalTok{,}\DataTypeTok{sheet =} \StringTok{"sueldos funcionarios"}\NormalTok{)}
\CommentTok{#alternativamente podemos especificar el número de orden de la hoja que deseamos levantar}
\NormalTok{Indices_Salario <-}\StringTok{ }\KeywordTok{read.xlsx}\NormalTok{(}\DataTypeTok{xlsxFile =} \StringTok{"../resultados/archivo_2_hojas.xlsx"}\NormalTok{,}\DataTypeTok{sheet =} \DecValTok{1}\NormalTok{)}
\NormalTok{Indices_Salario[}\DecValTok{1}\OperatorTok{:}\DecValTok{10}\NormalTok{,]}
\end{Highlighting}
\end{Shaded}

\begin{verbatim}
##             cuil anio mes funcionario_apellido funcionario_nombre
## 1  20-17692128-6 2019   1    RODRIGUEZ LARRETA    HORACIO ANTONIO
## 2  20-17735449-0 2019   1             SANTILLI        DIEGO CESAR
## 3  27-24483014-0 2019   1                ACUÑA      MARIA SOLEDAD
## 4  20-13872301-2 2019   1             ASTARLOA      GABRIEL MARIA
## 5  20-25641207-2 2019   1             AVOGADRO       ENRIQUE LUIS
## 6  27-13221055-7 2019   1            BOU PEREZ          ANA MARIA
## 7  27-13092400-5 2019   1                FREDA     MONICA BEATRIZ
## 8  20-17110752-1 2019   1         MACCHIAVELLI    EDUARDO ALBERTO
## 9  20-22293873-3 2019   1               MIGUEL       FELIPE OSCAR
## 10 20-14699669-9 2019   1               MOCCIA             FRANCO
##                                         repartición asignacion_por_cargo_i
## 1                                  Jefe de Gobierno               197745.8
## 2                          Vicejefatura de Gobierno               197745.8
## 3              Ministerio de Educación e Innovación               224516.6
## 4  Procuración General de la Ciudad de Buenos Aires               224516.6
## 5                             Ministerio de Cultura               224516.6
## 6                               Ministerio de Salud               224516.6
## 7  Sindicatura General de la Ciudad de Buenos Aires               224516.6
## 8          Ministerio de Ambiente y Espacio Público               224516.6
## 9                 Jefatura de Gabinete de Ministros               224516.6
## 10     Ministerio de Desarrollo Urbano y Transporte               224516.6
##    aguinaldo_ii total_salario_bruto_i_._ii observaciones
## 1             0                   197745.8              
## 2             0                   197745.8              
## 3             0                   224516.6              
## 4             0                   224516.6              
## 5             0                   224516.6              
## 6             0                   224516.6              
## 7             0                   224516.6              
## 8             0                   224516.6              
## 9             0                   224516.6              
## 10            0                   224516.6
\end{verbatim}

\bibliography{book.bib,packages.bib}


\end{document}
